\documentclass[12pt]{article}

\usepackage[T2A]{fontenc}
\usepackage[utf8]{inputenc}
\usepackage[english, russian]{babel}
\usepackage[sups]{XCharter}
\usepackage[vvarbb, uprightscript, charter, scaled=1.05]{newtxmath}
\usepackage{enumitem}
\usepackage{verbatim}
% \usepackage{caption}
% \captionsetup[figure]{skip=1pt}
\usepackage{microtype}
% \usepackage[style=numeric, sorting=none]{biblatex}
% \addbibresource{refs.bib}
% \usepackage{minted}
% \usepackage{fancyhdr}
% \usepackage{gensymb}
% \usepackage{booktabs}
% \usepackage{ntheorem}
% \usepackage{mathtools}
\usepackage{geometry}
% \usepackage{titling}  
\usepackage{indentfirst}
% \usepackage[normalem]{ulem}
% \useunder{\uline}{\ul}{}
\usepackage{graphicx}
\graphicspath{ {../vis/} }

\usepackage[table,xcdraw]{xcolor}
\usepackage{hyperref}
\hypersetup{
    colorlinks=true,
    linkcolor=blue,
    filecolor=magenta,      
    urlcolor=cyan,
}

\geometry{a4paper, textwidth=16cm, textheight=24cm}

\newcommand{\mpl}[2]{
    \begin{figure}[!h]
        \includegraphics[width=0.98\textwidth]{#1}
        \centering
        \caption{#2}
        \label{fig:#1}
     \end{figure}
}

\title{Отчет по заданию №3:\break Композиции алгоритмов для решения \break задачи регрессии}
\author{Васильев Руслан \and{ВМК МГУ, 317 группа}}

\begin{document}

\maketitle
\tableofcontents
\newpage
% \setcounter{secnumdepth}{0}
\section{Введение}
В заключительном практическом задании предлагается реализовать композиции алгоритмов машинного обучения и провести эксперименты, а также спроектировать веб-сервис для взаимодействия с моделью. Весь проект доступен в репозитории\footnote{\url{https://github.com/artnitolog/mmf_prac_2020_task_3}}. Данный отчет иллюстрирует результаты экспериментов с моделями на датасете данных о продажи недвижимости.

\section{Постановка задачи}

Итак, рассматривается задача регрессии с метрикой качества RMSE:
\begin{equation*}
    \operatorname{RMSE} = \sqrt{\frac{\sum_{i=1}^{N} (y_i - \hat{y}_i)}{N}},
\end{equation*}
где $N$~--- размер выборки, $y_i$~--- истинное значение целевой переменной на $i$-м объекте, $\hat{y}_i$~--- предсказанное.

Для решения реализованы две модели, представляющие собой ансамбли решающих деревьев: случайный лес и градиентный бустинг. Исследование алгоритмов включает в себя измерение функции ошибки и времени работы при варьировании гиперпараметров (порядок экспериментов соответствует стандартной настройке данных моделей).

\section{Эксперименты}
\subsection{Предобработка данных}
Исходные данные о недвижимости были разделены на обучение (80\%) и контроль (20\%, она же валидационная выборка). И здесь сразу учитывается особенность задачи. Хотя в задании отсутствует описание признаков и целевой переменной, можно с уверенностью предположить, что столбец \verb|date| связан со временем поступления данных (даты имеют небольшой диапазон 2014--2015, монотонно возрастают, дублируются, следуют сразу за \verb|ID|, а столбцы \verb|build_year| и \verb|renovation_year| с ними не связаны). По этой причине было бы некорректно перемешать выборку перед разделением на обучение и контроль~--- из-за утечки такая стратегия может дать ложную оценку качества моделей и привести к неправильным выводам. В качестве валидационной выборки берутся последние 20\% данных, соответствующие хронологическому порядку по столбцу \verb|date|.

\subsection{Случайный лес}
\subsubsection{Количество деревьев}
\subsubsection{Размерность подвыборки признаков для дерева}
\subsubsection{Глубина дерева}

\subsection{Градиентный бустинг}
\subsubsection{Количество деревьев и темп обучения}
\subsubsection{Глубина дерева}
\subsubsection{Размерность подвыборки признаков для дерева}
\section{Заключение}
\end{document}